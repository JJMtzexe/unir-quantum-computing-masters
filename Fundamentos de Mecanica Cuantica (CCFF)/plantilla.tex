\documentclass[11pt,a4paper,spanish]{book}
\usepackage{estilo_unir-1}
\usepackage{braket}

%---------------------------
%título del trabajo y autor
%---------------------------
\title{Actividad: Fundamentos de la mecánica cuántica}
\author{Francisco Javier Martínez Rodríguez, Ivan Dario Torres Garnic, Jeyson Antonio Flores Deras, José Juan Martínez Tapia}
\date{16 de febrero de 2026}
\profesor{Rodrigo Gil-Merino y Rubio}

%---------------------------
%marges
%---------------------------
%\usepackage[margin=1.9cm]{geometry}
%---------------------------
%---------------------------
%---------------------------
%---------------------------
\begin{document}
\renewcommand{\listfigurename}{Índice de figuras}
\renewcommand{\listtablename}{Índice de tablas}
\renewcommand{\contentsname}{Índice de contenidos}
\renewcommand{\figurename}{Figura}
\renewcommand{\tablename}{Tabla} 

\maketitle

\frontmatter
\tableofcontents
\listoffigures %atención, quitar si no hay figuras!!!
\listoftables %atención, quitar si no hay tablas!!!

\chapter{Resumen}

Este trabajo analizará la revolución conceptual que transformó la física entre finales del siglo XIX y principios del XX, cuando se empezaron a investigar y observar comportamientos que las reglas ``clásicas'' no podían explicar, como la famosa catástrofe ultravioleta, el efecto fotoeléctrico, la cuantización de la energía, entre otros.

El objetivo es profundizar en el contexto histórico y el razonamiento detrás de cada uno de los hitos que dieron lugar al nacimiento de la ``mecánica cuántica antigua''. Al concluir este documento, el lector deberá comprender más a fondo y a detalle la historia detrás de cada descubrimiento.

El trabajo concluye en que esta transición hacia la mecánica de ondas no solamente resolvió las crisis a explorar, sino que estableció las bases indispensables de la computación cuántica, tema que abordaremos en futuras asignaturas (fuera del alcance de este documento).\\

{\bf Palabras clave:} Radiación de cuerpo negro, Cuantización, Ecuación de Schrödinger, Mecánica ondulatoria.\\

\chapter{Abstract}

This document will analyze the conceptual revolution that transformed physics between the late 19th and early 20th centuries, when scientists began observing and investigating unknown phenomena which the ``classical'' rules at that time could not explain, such as the ultraviolet catastrophe, the photoelectric effect, the quantization of energy, amongst others.

The goal is to delve deeper into the historical context and the reasoning behind each one of these achievements that led to the origin of the ``old quantum theory''. Upon finishing this document, the reader should gain a deeper and more detailed understanding of the story behind each discovery.

This document concludes that the transition toward wave mechanics not only resolved the crises to be explored below, but also established the essential foundations of quantum computing, a topic we will address in future courses (out of this paper's scope).\\

{\bf Keywords:} Black-body radiation, Quantization, Schrödinger equation, Wave mechanics.\\

\mainmatter
\chapter{Introducción}

En la Introducción se debe resumir de forma esquemática pero suficientemente clara lo esencial de cada una de las partes del trabajo. La lectura de esta parte debe contextualizar perfectamente todo el trabajo y debe estar PLAGADA DE REFERENCIAS.\\

Las referencias NO están para rellenar. Son un TRIBUTO a las personas que hicieron en primer lugar una investigación o aportaron una idea, por tanto, se deben citar LOS TRABAJOS ORIGINALES DE LOS AUTORES, y no un libro de texto donde he visto que hablan de algo.\\

Es una parte muy importante de la memoria. Las ideas principales a transmitir son la identificación del problema a tratar, la justificación de su importancia, los objetivos generales a grandes rasgos y un adelanto de la contribución que esperas hacer.\\

A modo de guía, la Introducción debe contener estos tres apartados:
\begin{itemize}
\item Motivación / justificación del tema a tratar
\item Planteamiento del Trabajo
\item Estructura del Trabajo
\end{itemize}


ATENCIÓN:  Si queremos citar a alguien, por ejemplo porque vamos a hablar de Latex \citep{lamport1994} o porque, según las ideas de \cite{ackerman2017}, la liga de fútbol inglesa debe tener torneos de desempate, pues tenemos que hacerlo correctamente.



\chapter{Contexto y estado de la cuestión}\label{contexto}

En esta Sección~\ref{contexto} debemos demostrar que conocemos lo que se ha hecho en el ámbito que estamos desarrollando el Trabajo. En nuestro caso, que se ha buscado la bibliografía y referencias suficientes y que esas ideas se han volcado en el Trabajo en la línea de los objetivos que perseguimos o que queremos transmitir.


\chapter{Objetivos}

Esquematizar claramente los objetivos del Trabajo, las ideas que queremos demostrar. Podemos tener varios tipos de objetivos. Deben ordenarse claramente.

\chapter{Desarrollo del Trabajo}

Aquí desarrollaremos nuestro Trabajo. Contrastaremos la ideas entre varios autores y, si es posible, con las nuestras. Podemos incluir los subapartados que necesitemos.

\chapter{Conclusiones}

Las Conclusiones es otra parte muy IMPORTANTE de la memoria. Deben ser muy clara. Si es posible se pueden itemizar o, mejor, poner un párrafo por idea con un pequeño título ilustrativo.

\addcontentsline{toc}{chapter}{Bibliografía}
\begin{thebibliography}{a}
%%\bibitem{etiqueta} \textsc{Autores},\textit{nombre referencia.}Información addicional
\bibitem[Lamport(1994)]{lamport1994} Lamport, L. (1994) \emph{\LaTeX: a document preparation system}, Addison
Wesley, Massachusetts, 2nd ed.
\bibitem[Ackerman(2017)]{ackerman2017} Ackerman, E. (2017) Why the English Premier League Should Have Playoffs.  Balls.ie. 
\end{thebibliography}
%\bibliographystyle{plain} 
%\bibliography{bibliografia}


\appendix
\chapter{Apéndices}

Aquí se pueden poner desarrollos matemáticos engorrosos de los que se puede prescindir en el cuerpo principal de la memoria u otros añadidos que aportan información pero no encajan correctamente en las secciones anteriores.\\

Si no ya apéndices, quitar esta Sección

\end{document}





















