\documentclass[12pt,a4paper,spanish]{book}
\usepackage{fontspec}
\setmainfont[
    Path=shared/fonts/,
    Extension=.ttf,
    UprightFont=*-Regular,
    BoldFont=*-Bold,
    ItalicFont=*-Italic,
    BoldItalicFont=*-BoldItalic
]{Calibri} 
\usepackage{shared/estilo_unir-1}
\usepackage{braket}
\usepackage{setspace}
\onehalfspacing
\graphicspath{{fundamentos_de_mecanica_cuantica/images/}{images/}}

%---------------------------
%título del trabajo y autor
%---------------------------
\title{Actividad: Fundamentos de la mecánica cuántica}
\author{Francisco Javier Martínez Rodríguez, Ivan Dario Torres Garnic, Jeyson Antonio Flores Deras y José Juan Martínez Tapia}
\date{16 de febrero de 2026}
\profesor{Rodrigo Gil-Merino y Rubio}

%---------------------------
%marges
%---------------------------
%\usepackage[margin=1.9cm]{geometry}
%---------------------------
%---------------------------
%---------------------------
%---------------------------
\begin{document}
\renewcommand{\listfigurename}{Índice de figuras}
\renewcommand{\listtablename}{Índice de tablas}
\renewcommand{\contentsname}{Índice de contenidos}
\renewcommand{\figurename}{Figura}
\renewcommand{\tablename}{Tabla} 

\maketitle

\frontmatter
\tableofcontents
\listoffigures %atención, quitar si no hay figuras!!!
\listoftables %atención, quitar si no hay tablas!!!

\chapter{Resumen}

Este trabajo analizará la revolución conceptual que transformó la física entre finales del siglo XIX y principios del XX, cuando se empezaron a investigar y observar comportamientos que las reglas ``clásicas'' no podían explicar, como la famosa catástrofe ultravioleta, la cuantización de la energía~\cite{planckm1900_quantization_2}, el efecto fotoeléctrico~\cite{einsteina1905_photoelectric}, entre otros.

El objetivo es profundizar en el contexto histórico y el razonamiento detrás de cada uno de los hitos que dieron lugar al nacimiento de la ``mecánica cuántica antigua''. Al concluir este documento, el lector deberá comprender más a fondo y a detalle la historia detrás de cada descubrimiento.

El trabajo concluye en que esta transición hacia la mecánica de ondas no solamente resolvió las crisis a explorar, sino que estableció las bases indispensables de la computación cuántica, tema que abordaremos en futuras asignaturas (fuera del alcance de este documento).\\

{\bf Palabras clave:} Radiación de cuerpo negro, Cuantización, Ecuación de Schrödinger, Mecánica ondulatoria.\\

\chapter{Abstract}

This document will analyze the conceptual revolution that transformed physics between the late 19th and early 20th centuries, when scientists began observing and investigating unknown phenomena which the ``classical'' rules at that time could not explain, such as the ultraviolet catastrophe, the quantization of energy~\cite{planckm1900_quantization_2}, the photoelectric effect~\cite{einsteina1905_photoelectric}, amongst others.

The goal is to delve deeper into the historical context and the reasoning behind each one of these achievements that led to the origin of the ``old quantum theory''. Upon finishing this document, the reader should gain a deeper and more detailed understanding of the story behind each discovery.

This document concludes that the transition toward wave mechanics not only resolved the crises to be explored below, but also established the essential foundations of quantum computing, a topic we will address in future courses (out of this paper's scope).\\

{\bf Keywords:} Black-body radiation, Quantization, Schrödinger equation, Wave mechanics.\\

\mainmatter
\chapter{Introducción}

En la Introducción se debe resumir de forma esquemática pero suficientemente clara lo esencial de cada una de las partes del trabajo. La lectura de esta parte debe contextualizar perfectamente todo el trabajo y debe estar PLAGADA DE REFERENCIAS.\\

Las referencias NO están para rellenar. Son un TRIBUTO a las personas que hicieron en primer lugar una investigación o aportaron una idea, por tanto, se deben citar LOS TRABAJOS ORIGINALES DE LOS AUTORES, y no un libro de texto donde he visto que hablan de algo.\\

Es una parte muy importante de la memoria. Las ideas principales a transmitir son la identificación del problema a tratar, la justificación de su importancia, los objetivos generales a grandes rasgos y un adelanto de la contribución que esperas hacer.\\

A modo de guía, la Introducción debe contener estos tres apartados:
\begin{itemize}
\item Motivación / justificación del tema a tratar
\item Planteamiento del Trabajo
\item Estructura del Trabajo
\end{itemize}


ATENCIÓN:  Si queremos citar a alguien, por ejemplo porque vamos a hablar de Latex \citep{lamport1994} o porque, según las ideas de \cite{ackerman2017}, la liga de fútbol inglesa debe tener torneos de desempate, pues tenemos que hacerlo correctamente.



\chapter{Contexto y estado de la cuestión}\label{contexto}

En esta Sección~\ref{contexto} debemos demostrar que conocemos lo que se ha hecho en el ámbito que estamos desarrollando el Trabajo. En nuestro caso, que se ha buscado la bibliografía y referencias suficientes y que esas ideas se han volcado en el Trabajo en la línea de los objetivos que perseguimos o que queremos transmitir.


\chapter{Objetivos}

Esquematizar claramente los objetivos del Trabajo, las ideas que queremos demostrar. Podemos tener varios tipos de objetivos. Deben ordenarse claramente.

\chapter{Desarrollo del Trabajo}

Aquí desarrollaremos nuestro Trabajo. Contrastaremos la ideas entre varios autores y, si es posible, con las nuestras. Podemos incluir los subapartados que necesitemos.\\

\section{La Radiación del Cuerpo Negro: De lo Clásico a Planck.}

Lorem ipsum dolor sit amet, consectetur adipiscing elit. Sed do eiusmod tempor incididunt ut labore et dolore magna aliqua. Ut enim ad minim veniam, quis nostrud exercitation ullamco laboris nisi ut aliquip ex ea commodo consequat. Duis aute irure dolor in reprehenderit in voluptate velit esse cillum dolore eu fugiat nulla pariatur.

\section{El Fotón y la Estructura Nuclear: Einstein y Rutherford.}

\subsection{El Efecto Fotoeléctrico}
Según la teoría clásica, la luz es una perturbación periódica continua (u onda electromagnética) que se propaga en el espacio. Esta teoría predecía, por ejemplo, que en el efecto fotoeléctrico la energía de los electrones emitidos debería depender de la intensidad de la luz.
Sin embargo en el postulado del efecto fotoeléctrico ~\cite{einsteina1905_photoelectric} se demostró que la luz era de naturaleza corpuscular (formada por cuantos) así como la existencia de los fotones. A través de la experimentación se demostró que los electrones saltaban de forma instantánea y solo si la luz superaba una frecuencia umbral específica, independientemente de cuán intensa fuera la luz. Esta relación se expresa matemáticamente como:
\begin{equation}
    E_{\gamma} = h\nu = W_0 + E_{\text{cin}}
\end{equation}
Siendo que:
\begin{itemize}
\item $h\nu$ es la constante de Plank ($h$)~\cite{planckm1900_quantization_2} medida en Julios por segundo ($J \cdot s$) multiplicada por la frecuencia de la radiación ($\nu$) medida en segundos a la inversa ($s^{-1}$), generando que el resultado de la expresión sea en Julios ($J$).
\item $E_{\gamma}$ Es la energía del fotón incidente.
\item $W_0$ es la energía de extracción.
\item $E_{\text{cin}}$ es la energía cinética máxima de los electrones emitidos.
\end{itemize}
Esta formulación tiene la restricción de que $W_0$ debe ser mayor a $h\nu$ para que se produzca la emisión de electrones.

Teniendo en cuenta la estructura de la ecuación anterior, es posible calcular la frecuencia umbral necesaria para que se produzca el efecto fotoeléctrico, despejando la expresión para obtener $\nu$:
\begin{equation}
    \nu = \frac{W_0 + E_{\text{cin}}}{h}
\end{equation}
\section{El Átomo de Bohr-Sommerfeld y los Espectros Atómicos.}

Lorem ipsum dolor sit amet, consectetur adipiscing elit. Sed do eiusmod tempor incididunt ut labore et dolore magna aliqua. Ut enim ad minim veniam, quis nostrud exercitation ullamco laboris nisi ut aliquip ex ea commodo consequat. Duis aute irure dolor in reprehenderit in voluptate velit esse cillum dolore eu fugiat nulla pariatur.

\section{El Modelo de Schrödinger: Mecánica Ondulatoria.}

Esta sección explorará el contexto histórico y el razonamiento que dieron lugar a la famosa ecuación de ondas de Erwin Schrödinger~\cite{schrodinger1926}. Partiremos del acierto del modelo de Bohr respecto a que los niveles de energía del átomo podían cuantificarse (no obstante, aún sin fundamentos), hasta la construcción de la mecánica ondulatoria y orbital, y los trabajos subsecuentes.

\subsection{¿Qué pendientes dejaron los modelos de Bohr-Sommerfeld?}

Primero, recordemos que el modelo de \cite{rutherford1911} estableció que los electrones orbitan alrededor de un núcleo de carga positiva, y según la electrodinámica clásica, estos deberían de perder energía en forma de radiación contínua y caer al núcleo en cuestion de centésimas de nanosegundos; cosa que diverge de la realidad.

Bohr encontró una alternativa en 1913, postulando que los átomos tienen varios estados ``estacionarios'' en los que el electrón no produce radiación, y las órbitas solo pueden tener un momento angular múltiplo de $\hbar$~\cite{bohr1913}:

\begin{equation}
L = n \hbar
\label{eq:angular_momentum_rutherford}
\end{equation}

Con estos postulados obtuvo los niveles de energía del hidrógeno ($E_n = -13.6/n^2$ eV) y explicó su espectro de emisión con notable precisión.

Posteriormente, \cite{sommerfeld1916} extendió el modelo permitiendo órbitas elípticas e introduciendo los números cuánticos azimutal $\ell$ y magnético $m$ mediante reglas de cuantización más generales:

\begin{equation}
E_n = -\frac{Z^2 e^4 m_e}{32\pi^2 \varepsilon_0^2 \hbar^2} \cdot \frac{1}{n^2}, \quad L = \ell\hbar, \quad L_z = m\hbar
\label{eq:sommerfeld}
\end{equation}

donde $\ell$ describe la forma de la órbita (con valores $0, 1, \ldots, n-1$) y $m$ su orientación espacial (con valores $-\ell, \ldots, +\ell$). Es importante recalcar que la energía $E_n$ seguía dependiendo exclusivamente de $n$, pues los números $\ell$ y $m$ no aparecían en la ecuación~\ref{eq:sommerfeld}. Esto implica que órbitas con formas y orientaciones distintas tienen la misma energía, algo que no coincidía con experimentos como el efecto Zeeman y la estructura fina, que mostraban desdoblamientos en las líneas espectrales que solo podían explicarse si la energía también dependía de $\ell$ y $m$.

Al igual que en el modelo de Bohr, los valores permitidos de $\ell$ y $m$ se imponen mediante reglas de cuantización sin una justificación teórica de fondo. Seguían siendo restricciones que funcionaban, pero que aún nadie podía derivar.

\subsubsection{Algunas limitaciones adicionales a lo anterior son:}

\begin{itemize}
    \item El modelo fallaba al describir átomos con más de un electrón.
    \item No se explicaba el mecanismo por el cual un electrón transiciona entre niveles de energía.
    \item El electrón se seguía tratando como una partícula puntual en una trayectoria definida.
\end{itemize}

Uno de los cuestionamientos más notables y conocidos fue el del propio Rutherford, quien en una carta dirigida a Bohr señaló: ``¿cómo decide un electrón a qué frecuencia va a vibrar cuando pasa de un estado estacionario a otro? Pareciera que se debe asumir que el electrón sabe de antemano dónde se va a detener''~\cite{rutherford1913carta}. \textbf{Hacía falta, por tanto, cambiar el panorama completo.}

\subsection{De Broglie y su razonamiento de la dualidad onda-partícula}

En 1905, Albert Einstein demostró que la luz—hasta entonces considerada exclusivamente como  una onda electromagnética—también se comportaba ocasionalmentecomo partícula (el fotón $\gamma$ ), cuya energía viene dada por $E = h\nu$~\cite{einsteina1905_photoelectric}. En 1924, Louis de Broglie reflejó este razonamiento en su tesis doctoral: si las ondas pueden comportarse como partículas, ¿pueden las partículas comportarse como ondas?~\cite{debroglie1924}
 
\subsubsection{De Broglie propuso que toda partícula con momento $p$ tiene asociada una longitud de onda $\lambda$:}

\begin{equation}
\lambda = \frac{h}{p} = \frac{h}{m_e v}
\label{eq:debroglie}
\end{equation}

Esta hipótesis tuvo una consecuencia inmediata sobre el modelo de Bohr. Si el electrón es una onda, su órbita alrededor del núcleo debe contener un número entero de longitudes de onda para que la onda sea estacionaria—es decir, para que al completar una vuelta, la onda ``cierre'' sobre sí misma sin destruirse por interferencia destructiva. 

\begin{figure}[h]
    \centering
    \includegraphics[width=0.8\textwidth]{ondas_estacionarias_debroglie.png}
    \caption{Ondas estacionarias de de Broglie en órbitas de radio 
    creciente. Fuente: \cite{elert2024}.}
    \label{fig:debroglie_ondas}
\end{figure}

\noindent{Esta condición se expresa como:}

\begin{equation}
2\pi r = n\lambda = n\frac{h}{m_e v}
\label{eq:onda_estacionaria}
\end{equation}

\noindent{Al despejar, se obtiene directamente:}

\begin{equation}
m_e v r = n\hbar
\label{eq:bohr_desde_debroglie}
\end{equation}

que es precisamente la condición de cuantización del momento angular que Bohr había postulado como restricción once años antes. Lo que en el modelo de Bohr era un hecho sin precedentes, en el razonamiento de de Broglie se convertía en una consecuencia natural de la naturaleza ondulatoria del electrón.

Sin embargo, mientras la onda del electrón se seguía tratando como unidimensional, envuelta alrededor de una trayectoria circular, un electrón real existe en tres dimensiones, y su comportamiento ondulatorio debería describirse mediante una ecuación de ondas tridimensional—análoga a las ecuaciones de Maxwell para la luz. Como señala el material del curso, ``Schrödinger aprovecha la idea de De Broglie de asociar una onda a cada partícula, y desarrolla una ecuación de ondas que deben cumplir todas las partículas que pudiesen representarse mediante una función de onda $\Psi$''~\cite{tema2}.

\subsection{La ecuación de onda de Schrödinger}

Ante las limitaciones de la teoría cuántica antigua, surgieron dos enfoques para construir una mecánica cuántica completa. En 1925, Werner Heisenberg propuso trabajar directamente con matrices de magnitudes observables, dando lugar a la mecánica matricial~\cite{heisenberg1925}. De forma independiente, Erwin Schrödinger partió de la hipótesis de de Broglie y desarrolló una ecuación de ondas que toda partícula debía satisfacer~\cite{schrodinger1926}. Aunque los dos caminos parecían muy distintos, el propio Schrödinger demostró que eran matemáticamente equivalentes~\cite{tema2}. Como ejemplo, el momento angular $L_z$ en ambas representaciones se expresa como:

\begin{equation}
\text{Heisenberg: } L_z = Q_x P_y - Q_y P_x
\label{eq:heisenberg_lz}
\end{equation}

\begin{equation}
\text{Schrödinger: } L_z = -i\hbar\left(x\frac{\partial}{\partial y} - y\frac{\partial}{\partial x}\right)
\label{eq:schrodinger_lz}
\end{equation}

En ambos casos, la estructura es la misma: posición en $x$ por momento en $y$, menos posición en $y$ por momento en $x$. Heisenberg representa la posición ($Q$) y el momento ($P$) como matrices, mientras que Schrödinger representa la posición como la coordenada $x$ y el momento como el operador diferencial $-i\hbar\,\partial/\partial x$.

\noindent Centrando la atención en el enfoque ondulatorio, la ecuación de Schrödinger en su forma completa es:

\begin{equation}
i\hbar\frac{\partial}{\partial t}\Psi(\mathbf{r},t) = -\frac{\hbar^2}{2m}\nabla^2\Psi(\mathbf{r},t) + U(\mathbf{r})\Psi(\mathbf{r},t)
\label{eq:schrodinger_td}
\end{equation}

donde $\Psi(\mathbf{r},t)$ es la función de onda de la partícula, $m$ su masa, $U(\mathbf{r})$ el potencial al que está sometida, y $\nabla^2$ es el operador laplaciano. El lado izquierdo describe cómo evoluciona $\Psi$ en el tiempo. En el lado derecho, el primer término representa la energía cinética y el segundo la energía potencial.

Cuando el potencial no depende del tiempo—como ocurre con la interacción coulombiana entre el electrón y el núcleo—la función de onda puede separarse en dos partes independientes:

\begin{equation}
\Psi(\mathbf{r},t) = \psi(\mathbf{r})\,\phi(t)
\label{eq:separacion}
\end{equation}

La parte temporal tiene una solución directa, $\phi(t) = e^{-iEt/\hbar}$, que al calcular la densidad de probabilidad $|\Psi|^2 = |\psi|^2 \cdot |\phi|^2$ desaparece, ya que $|e^{-iEt/\hbar}|^2 = 1$. Esto significa que la distribución de probabilidad no cambia con el tiempo, lo cual justifica el nombre de ``estados estacionarios'' que Bohr había introducido en 1913~\cite{bohr1913}: ahora, en vez de ser un postulado, es una consecuencia matemática de la ecuación.

\noindent La parte espacial satisface la ecuación de Schrödinger independiente del tiempo:

\begin{equation}
E\psi(\mathbf{r}) = -\frac{\hbar^2}{2m}\nabla^2\psi(\mathbf{r}) + U(\mathbf{r})\psi(\mathbf{r})
\label{eq:schrodinger_ti}
\end{equation}

Esta es la ecuación que se resuelve para obtener los niveles de energía $E$ y las funciones de onda $\psi(\mathbf{r})$ permitidas. A diferencia de los modelos de Bohr y Sommerfeld, la cuantización no se impone: emerge naturalmente de las condiciones de contorno que debe satisfacer $\psi$ (ser finita, continua y normalizable). En el modelo de Schrödinger, las órbitas del electrón son sustituidas por \textit{orbitales} espaciales que representan la probabilidad de encontrar al electrón en un punto del espacio~\cite{tema2}.


\subsection{Solución para el hidrógeno: números cuánticos, orbitales y nubes de probabilidad}

Para aplicar la fórmula anterior al átomo del hidrógeno, se toma en cuenta que la energía potencial $U(r)$  es la de Coulomb: $-e^2/(4\pi\varepsilon_0 r)$. Ya que esto depende únicamente de la distancia $r$ al núcleo, podemos restructurar en coordenadas esféricas $(r, \theta, \phi)$, donde la función de onda puede representarse en dos partes:

\begin{equation}
\psi(r,\theta,\phi) = R_{n,\ell}(r) \cdot Y_\ell^{m}(\theta,\phi)
\label{eq:separacion_esferica}
\end{equation}

$R(r)$, describe cómo varía la probabilidad con la distancia al núcleo. La otra, $Y(\theta,\phi)$, describe cómo varía con la dirección.

Como ecuación matemática, la ecuación de Schrödinger admite soluciones para cualquier valor de energía. Sin embargo, la mayoría de estas soluciones no describen un electrón real: algunas divergen al infinito, otras presentan discontinuidades, y otras dan probabilidades totales distintas de 1. Al descartar estas soluciones no físicas y quedarnos únicamente con aquellas que son finitas, continuas y normalizables, solo sobreviven soluciones para ciertos valores discretos de energía y momento angular. Estos valores quedan etiquetados por tres números cuánticos:

\begin{itemize}
\item $n = 1, 2, 3, \ldots$ (principal): determina la energía del electrón, $E_n = -13.6/n^2$ eV. Surge de la parte radial $R(r)$ de la función de onda.
\item $\ell = 0, 1, \ldots, n-1$ (azimutal): determina el momento angular orbital $L = \sqrt{\ell(\ell+1)}\,\hbar$ y define las subcapas $s, p, d, f$. Surge de la parte angular $Y(\theta, \phi)$.
\item $m_\ell = -\ell, \ldots, +\ell$ (magnético): determina la orientación del momento angular en el espacio, $L_z = m_\ell\hbar$. Explica el efecto Zeeman. También surge de la parte angular.
\end{itemize}

Es decir, cada número cuántico aparece porque la función de onda, al tener que cumplir condiciones físicas en una geometría esférica, solo ``encaja'' para ciertos valores.

\begin{figure}[h]
    \centering
    \includegraphics[width=0.7\textwidth]{orbitales_schrodinger_born.png}
    \caption{Orbitales atómicos de Schrödinger-Born, etiquetados por $(n, \ell, m)$. Fuente: \cite{malgieri2022}.}
    \label{fig:orbitales}
\end{figure}

La interpretación física de $\psi$ fue proporcionada por \cite{born1926}: $|\psi(\mathbf{r})|^2$ representa la densidad de probabilidad de encontrar al electrón en la posición $\mathbf{r}$. Las ``órbitas'' definidas del modelo de Bohr se transforman en nubes de probabilidad tridimensionales—los orbitales. Los orbitales $s$ ($\ell = 0$) son esféricos, los $p$ ($\ell = 1$) tienen forma bilobular, y los $d$ ($\ell = 2$) presentan formas cuadrilobulares.

Un cuarto número cuántico, el espín ($m_s = \pm 1/2$), fue introducido por \cite{pauli1925} para explicar la estructura fina de los espectros. Pauli además enunció su principio de exclusión: dos electrones en un átomo no pueden compartir los mismos cuatro números cuánticos. Esto limita cada orbital a dos electrones con espines opuestos y determina la capacidad de cada capa electrónica: $2n^2$ electrones. El llenado progresivo de orbitales según este principio reproduce la estructura de la tabla periódica: los bloques $s$, $p$, $d$ y $f$ corresponden directamente a los valores de $\ell$.

\subsection{¿Qué pendientes dejó el trabajo de Schrödinger?}

A pesar de representar un salto importante respecto a la teoría cuántica antigua, el modelo de Schrödinger presenta limitaciones que el propio material del curso identifica~\cite{tema2}:

\begin{itemize}
\item No incorpora el espín electrónico de forma natural. El espín fue introducido por Pauli como un postulado adicional~\cite{pauli1925}, pero no emerge de la ecuación de Schrödinger.
\item No contempla efectos relativistas. Para electrones en átomos pesados, donde las velocidades son una fracción significativa de la velocidad de la luz, la ecuación pierde precisión ya que se da por hecho que la masa del electrón es constante.
\item No explica el mecanismo por el cual un electrón salta de un nivel energético a otro.
\end{itemize}

Las dos primeras limitaciones fueron resueltas por Paul Dirac en 1928, quien formuló una ecuación que unifica la mecánica cuántica con la relatividad especial. De esta ecuación, el espín y la existencia de la antimateria emergen como consecuencias naturales de la teoría, de la misma forma en que los números cuánticos $n$, $\ell$ y $m$ emergen de la ecuación de Schrödinger.

No obstante, la mecánica ondulatoria de Schrödinger estableció un cambio de paradigma que permanece vigente: los electrones no son partículas puntuales en órbitas definidas, sino distribuciones de probabilidad descritas por funciones de onda. Los números cuánticos no son postulados arbitrarios, sino consecuencias matemáticas de las condiciones de contorno. Y la estructura de la materia—desde el átomo de hidrógeno hasta la tabla periódica—emerge de una sola ecuación diferencial y un principio de exclusión. Estas ideas constituyen los cimientos sobre los que se construye la computación cuántica, tema que se abordará en asignaturas posteriores de este máster.

\chapter{Conclusiones}

Las Conclusiones es otra parte muy IMPORTANTE de la memoria. Deben ser muy clara. Si es posible se pueden itemizar o, mejor, poner un párrafo por idea con un pequeño título ilustrativo.

\addcontentsline{toc}{chapter}{Bibliografía}
\begin{thebibliography}{a}

%%\bibitem{etiqueta} \textsc{Autores},\textit{nombre referencia.}Información addicional

\bibitem[(Bohr, 1913)]{bohr1913} Bohr, N. (1913). On the constitution of atoms and molecules. \emph{Philosophical Magazine}, \emph{26}(151).

\bibitem[Born(1926)]{born1926} Born, M. (1926). Zur Quantenmechanik der Stoßvorgänge [Sobre la mecánica cuántica de los procesos de colisión]. \emph{Zeitschrift für Physik}, \emph{37}(12).

\bibitem[(de Broglie, 1924)]{debroglie1924} de Broglie, L. (1924). Recherches sur la théorie des quanta [Investigaciones sobre la teoría de los cuantos] (Tesis doctoral). Université de Paris.

\bibitem[(Einstein, 1905)]{einsteina1905_photoelectric} Einstein, A. (1905). Über einen die Erzeugung und Verwandlung des Lichtes betreffenden heuristischen Gesichtspunkt [Sobre un punto de vista heurístico relativo a la producción y transformación de la luz]. \emph{Annalen der Physik}.

\bibitem[(Elert, 2024)]{elert2024} Elert, G. (2024). Atomic Models. \emph{The Physics Hypertextbook}. https://physics.info/atomic-models/

\bibitem[(Gil-Merino, 2026)]{tema2} Gil-Merino, R. (2026). Tema 2: Teoría de la Radiación. \emph{Fundamentos de la Mecánica Cuántica}, Universidad Internacional de La Rioja (UNIR).

\bibitem[(Heisenberg, 1925)]{heisenberg1925} Heisenberg, W. (1925). Über quantentheoretische Umdeutung kinematischer und mechanischer Beziehungen [Sobre la reinterpretación cuántico-teórica de relaciones cinemáticas y mecánicas]. \emph{Zeitschrift für Physik}, \emph{33}(1).

\bibitem[(Malgieri, 2022)]{malgieri2022} Malgieri, M. (2022). What Schrödinger's cat tells us about the relationship between physics and philosophy. \emph{ResearchGate}. https://www.researchgate.net/publication/381897611

\bibitem[Pauli(1925)]{pauli1925} Pauli, W. (1925). Über den Zusammenhang des Abschlusses der Elektronengruppen im Atom mit der Komplexstruktur der Spektren [Sobre la conexión entre la completitud de los grupos de electrones en el átomo con la estructura compleja de los espectros]. \emph{Zeitschrift für Physik}, \emph{31}(1).

\bibitem[(Planck, 1900)]{planckm1900_quantization_2} Planck, M. (1900). Zur Theorie des Gesetzes der Energieverteilung im Normalspectrum [Sobre la teoría de la ley de distribución de energía en el espectro normal]. \emph{Verhandlungen der Deutschen Physikalischen Gesellschaft}.

\bibitem[Rutherford(1911)]{rutherford1911} Rutherford, E. (1911). The scattering of $\alpha$ and $\beta$ particles by matter and the structure of the atom. \emph{Philosophical Magazine}, \emph{21}(125).

\bibitem[(Rutherford, 1913)]{rutherford1913carta} Rutherford, E. (1913). Carta a N. Bohr. Reproducida en \emph{Physics of the Atom}. Publicada en \emph{Proceedings of the Physical Society}, \emph{78}(6).

\bibitem[(Schrödinger, 1926)]{schrodinger1926} Schrödinger, E. (1926). Quantisierung als Eigenwertproblem [Cuantización como problema de valores propios]. \emph{Annalen der Physik}.

\bibitem[Sommerfeld(1916)]{sommerfeld1916} Sommerfeld, A. (1916). Zur Quantentheorie der Spektrallinien [Sobre la teoría cuántica de las líneas espectrales]. \emph{Annalen der Physik}, \emph{356}(17).

\end{thebibliography}
%\bibliographystyle{plain} 
%\bibliography{bibliografia}


\appendix
\chapter{Apéndices}

El modelo de \cite{rutherford1911} estableció que los electrones orbitan alrededor de un núcleo de carga positiva. Sin embargo, según las leyes de la electrodinámica clásica, un electrón en movimiento orbital debería emitir radiación de forma continua, perder energía, y caer en espiral hacia el núcleo. \textbf{Esto, evidentemente, no ocurre.}

\begin{figure}[h]
    \centering
    \includegraphics[width=0.8\textwidth, height=0.2\textwidth]{espectroscopia_emision_h.jpg}
    \caption{Espectro de emisión del hidrógeno.}
    \label{fig:espectro_h}
\end{figure}

Asimismo, Bohr, al observar que las frecuencias del espectro de emisión del hidrógeno (Figura~\ref{fig:espectro_h}) no se podían explicar si se consideraba a la energía del electrón (y por ende, radiación) como un rango continuo, determinó que las líneas espectrales correspondían a la radiación emitida cuando un electrón transicionaba entre dos estados de energía discretos indicados por el número cuántico $n$~\cite{bohr1913}. La energía de estos estados viene dada:

\begin{equation}
E_n = -\frac{E_0}{n^2}, \quad \textit{donde } E_0 = \frac{e^4 m_e}{8\varepsilon_0^2 h^2} = 13.6 \text{ eV}
\label{eq:energia_bohr}
\end{equation}

$E_0$ es la energía del estado fundamental del átomo de hidrógeno, que resulta de combinar la energía cinética y potencial coulombiana del electrón con la condición de cuantización del 
momento angular (el momento angular se expresa por $L = n\hbar$; es decir, por cuantos de $\hbar$). La energía del fotón emitido durante la transición entre dos estados estacionarios es:

\begin{equation}
E_{\textit{fotón}} = E_i - E_f = h\nu
\end{equation}

Donde:
\begin{itemize}
\item $E_i$ : energía del estado inicial del electrón, donde $E_i > E_f$
\item $E_f$ : energía del estado final del electrón.
\item $h = 6.626 \times 10^{-34}$ J $\cdot$ s : constante de Planck.
\item $\nu$ : frecuencia del fotón emitido.
\end{itemize}

\end{document}