\documentclass[12pt,a4paper,spanish]{book}

\usepackage{shared/estilo_unir-1}
\usepackage{braket}

%---------------------------
%título del trabajo y autor
%---------------------------
\title{Actividad: Fundamentos de la mecánica cuántica}
\author{Francisco Javier Martínez Rodríguez, Ivan Dario Torres Garnic, Jeyson Antonio Flores Deras y José Juan Martínez Tapia}
\date{16 de febrero de 2026}
\profesor{Rodrigo Gil-Merino y Rubio}

%---------------------------
%marges
%---------------------------
%\usepackage[margin=1.9cm]{geometry}
%---------------------------
%---------------------------
%---------------------------
%---------------------------
\begin{document}
\renewcommand{\listfigurename}{Índice de figuras}
\renewcommand{\listtablename}{Índice de tablas}
\renewcommand{\contentsname}{Índice de contenidos}
\renewcommand{\figurename}{Figura}
\renewcommand{\tablename}{Tabla} 

\maketitle

\frontmatter
\tableofcontents
\listoffigures %atención, quitar si no hay figuras!!!
\listoftables %atención, quitar si no hay tablas!!!

\chapter{Resumen}

Este trabajo analizará la revolución conceptual que transformó la física entre finales del siglo XIX y principios del XX, cuando se empezaron a investigar y observar comportamientos que las reglas ``clásicas'' no podían explicar, como la famosa catástrofe ultravioleta, la cuantización de la energía~\cite{planckm1900_quantization_2}, el efecto fotoeléctrico~\cite{einsteina1905_photoelectric}, entre otros.

El objetivo es profundizar en el contexto histórico y el razonamiento detrás de cada uno de los hitos que dieron lugar al nacimiento de la ``mecánica cuántica antigua''. Al concluir este documento, el lector deberá comprender más a fondo y a detalle la historia detrás de cada descubrimiento.

El trabajo concluye en que esta transición hacia la mecánica de ondas no solamente resolvió las crisis a explorar, sino que estableció las bases indispensables de la computación cuántica, tema que abordaremos en futuras asignaturas (fuera del alcance de este documento).\\

{\bf Palabras clave:} Radiación de cuerpo negro, Cuantización, Ecuación de Schrödinger, Mecánica ondulatoria.\\

\chapter{Abstract}

This document will analyze the conceptual revolution that transformed physics between the late 19th and early 20th centuries, when scientists began observing and investigating unknown phenomena which the ``classical'' rules at that time could not explain, such as the ultraviolet catastrophe, the quantization of energy~\cite{planckm1900_quantization_2}, the photoelectric effect~\cite{einsteina1905_photoelectric}, amongst others.

The goal is to delve deeper into the historical context and the reasoning behind each one of these achievements that led to the origin of the ``old quantum theory''. Upon finishing this document, the reader should gain a deeper and more detailed understanding of the story behind each discovery.

This document concludes that the transition toward wave mechanics not only resolved the crises to be explored below, but also established the essential foundations of quantum computing, a topic we will address in future courses (out of this paper's scope).\\

{\bf Keywords:} Black-body radiation, Quantization, Schrödinger equation, Wave mechanics.\\

\mainmatter
\chapter{Introducción}

En la Introducción se debe resumir de forma esquemática pero suficientemente clara lo esencial de cada una de las partes del trabajo. La lectura de esta parte debe contextualizar perfectamente todo el trabajo y debe estar PLAGADA DE REFERENCIAS.\\

Las referencias NO están para rellenar. Son un TRIBUTO a las personas que hicieron en primer lugar una investigación o aportaron una idea, por tanto, se deben citar LOS TRABAJOS ORIGINALES DE LOS AUTORES, y no un libro de texto donde he visto que hablan de algo.\\

Es una parte muy importante de la memoria. Las ideas principales a transmitir son la identificación del problema a tratar, la justificación de su importancia, los objetivos generales a grandes rasgos y un adelanto de la contribución que esperas hacer.\\

A modo de guía, la Introducción debe contener estos tres apartados:
\begin{itemize}
\item Motivación / justificación del tema a tratar
\item Planteamiento del Trabajo
\item Estructura del Trabajo
\end{itemize}


ATENCIÓN:  Si queremos citar a alguien, por ejemplo porque vamos a hablar de Latex \citep{lamport1994} o porque, según las ideas de \cite{ackerman2017}, la liga de fútbol inglesa debe tener torneos de desempate, pues tenemos que hacerlo correctamente.



\chapter{Contexto y estado de la cuestión}\label{contexto}

En esta Sección~\ref{contexto} debemos demostrar que conocemos lo que se ha hecho en el ámbito que estamos desarrollando el Trabajo. En nuestro caso, que se ha buscado la bibliografía y referencias suficientes y que esas ideas se han volcado en el Trabajo en la línea de los objetivos que perseguimos o que queremos transmitir.


\chapter{Objetivos}

Esquematizar claramente los objetivos del Trabajo, las ideas que queremos demostrar. Podemos tener varios tipos de objetivos. Deben ordenarse claramente.

\chapter{Desarrollo del Trabajo}

Aquí desarrollaremos nuestro Trabajo. Contrastaremos la ideas entre varios autores y, si es posible, con las nuestras. Podemos incluir los subapartados que necesitemos.\\

\section{La Radiación del Cuerpo Negro: De lo Clásico a Planck.}

Lorem ipsum dolor sit amet, consectetur adipiscing elit. Sed do eiusmod tempor incididunt ut labore et dolore magna aliqua. Ut enim ad minim veniam, quis nostrud exercitation ullamco laboris nisi ut aliquip ex ea commodo consequat. Duis aute irure dolor in reprehenderit in voluptate velit esse cillum dolore eu fugiat nulla pariatur.

\section{El Fotón y la Estructura Nuclear: Einstein y Rutherford.}

Lorem ipsum dolor sit amet, consectetur adipiscing elit. Sed do eiusmod tempor incididunt ut labore et dolore magna aliqua. Ut enim ad minim veniam, quis nostrud exercitation ullamco laboris nisi ut aliquip ex ea commodo consequat. Duis aute irure dolor in reprehenderit in voluptate velit esse cillum dolore eu fugiat nulla pariatur.

\section{El Átomo de Bohr-Sommerfeld y los Espectros Atómicos.}

Lorem ipsum dolor sit amet, consectetur adipiscing elit. Sed do eiusmod tempor incididunt ut labore et dolore magna aliqua. Ut enim ad minim veniam, quis nostrud exercitation ullamco laboris nisi ut aliquip ex ea commodo consequat. Duis aute irure dolor in reprehenderit in voluptate velit esse cillum dolore eu fugiat nulla pariatur.

\section{El Modelo de Schrödinger: Mecánica Ondulatoria.}

A continuación, se expondrá el contexto histórico y el razonamiento que dieron lugar a la ecuación de ondas de Erwin Schrödinger~\cite{schrodinger1926}. El recorrido parte del acierto del modelo de Bohr respecto a la cuantización de los niveles de energía del átomo—pero sin ofrecer una justificación teórica de por qué era así—y llega hasta la formulación de la mecánica ondulatoria y los trabajos que se derivaron de ella.

\subsection{¿Qué pendientes dejaron los modelos de Bohr-Sommerfeld?}

\subsection{De Broglie y su razonamiento de la dualidad onda-partícula}

\subsection{La ecuación de onda de Schrödinger}

\subsection{Solución para el hidrógeno: los números cuánticos}

\subsection{Orbitales, nubes de probabilidad y el principio de exclusión}

\subsection{¿Qué pendientes dejó el trabajo de Schrödinger?}

\chapter{Conclusiones}

Las Conclusiones es otra parte muy IMPORTANTE de la memoria. Deben ser muy clara. Si es posible se pueden itemizar o, mejor, poner un párrafo por idea con un pequeño título ilustrativo.

\addcontentsline{toc}{chapter}{Bibliografía}
\begin{thebibliography}{a}

%%\bibitem{etiqueta} \textsc{Autores},\textit{nombre referencia.}Información addicional

\bibitem[(Bohr, 1913)]{bohr1913} Bohr, N. (1913). On the constitution of atoms and molecules. \emph{Philosophical Magazine}, \emph{26}(151).

\bibitem[(Einstein, 1905)]{einsteina1905_photoelectric} Einstein, A. (1905). Über einen die Erzeugung und Verwandlung des Lichtes betreffenden heuristischen Gesichtspunkt [Sobre un punto de vista heurístico relativo a la producción y transformación de la luz]. \emph{Annalen der Physik}.

\bibitem[(Planck, 1900)]{planckm1900_quantization_2} Planck, M. (1900). Zur Theorie des Gesetzes der Energieverteilung im Normalspectrum [Sobre la teoría de la ley de distribución de energía en el espectro normal]. \emph{Verhandlungen der Deutschen Physikalischen Gesellschaft}.

\bibitem[Rutherford (1911)]{rutherford1911} Rutherford, E. (1911). The scattering of $\alpha$ and $\beta$ particles by matter and the structure of the atom. \emph{Philosophical Magazine}, \emph{21}(125).

\bibitem[(Schrödinger, 1926)]{schrodinger1926} Schrödinger, E. (1926). Quantisierung als Eigenwertproblem [Cuantización como problema de valores propios]. \emph{Annalen der Physik}.

\end{thebibliography}
%\bibliographystyle{plain} 
%\bibliography{bibliografia}


\appendix
\chapter{Apéndices}

El modelo de \cite{rutherford1911} estableció que los electrones orbitan alrededor de un núcleo de carga positiva. Sin embargo, según las leyes de la electrodinámica clásica, un electrón en movimiento orbital debería emitir radiación de forma continua, perder energía, y caer en espiral hacia el núcleo. \textbf{Esto, evidentemente, no ocurre.}

\begin{figure}[h]
    \centering
    \includegraphics[width=0.8\textwidth, height=0.2\textwidth]{fundamentos_de_mecanica_cuantica/images/espectroscopia_emision_h.jpg}
    \caption{Espectro de emisión del hidrógeno.}
    \label{fig:espectro_h}
\end{figure}

Asimismo, Bohr, al observar que las frecuencias del espectro de emisión del hidrógeno (Figura~\ref{fig:espectro_h}) no se podían explicar si se consideraba a la energía del electrón (y por ende, radiación) como un rango continuo, determinó que las líneas espectrales correspondían a la radiación emitida cuando un electrón transicionaba entre dos estados de energía discretos indicados por el número cuántico $n$~\cite{bohr1913}. La energía de estos estados viene dada:

\begin{equation}
E_n = -\frac{E_0}{n^2}, \quad \textit{donde } E_0 = \frac{e^4 m_e}{8\varepsilon_0^2 h^2} = 13.6 \text{ eV}
\label{eq:energia_bohr}
\end{equation}

$E_0$ es la energía del estado fundamental del átomo de hidrógeno, que resulta de combinar la energía cinética y potencial coulombiana del electrón con la condición de cuantización del 
momento angular (el momento angular se expresa por $L = n\hbar$; es decir, por cuantos de $\hbar$). La energía del fotón emitido durante la transición entre dos estados estacionarios es:

\begin{equation}
E_{\textit{fotón}} = E_i - E_f = h\nu
\end{equation}

Donde:
\begin{itemize}
\item $E_i$ : energía del estado inicial del electrón, donde $E_i > E_f$
\item $E_f$ : energía del estado final del electrón.
\item $h = 6.626 \times 10^{-34}$ J $\cdot$ s : constante de Planck.
\item $\nu$ : frecuencia del fotón emitido.
\end{itemize}

\end{document}





















